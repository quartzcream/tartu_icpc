% !begin_codebook
\section{Factsheet}
\includegraphics[scale=0.8]{comb.pdf}

\begin{itemize}
\item \textbf{Fermat's little theorem.} Let $p$ be prime. Then, for each integer $a$:
  \begin{equation*}
    a^{p - 1} \equiv 1 \pmod{p}.
  \end{equation*}
  Thus:
  \begin{equation*}
    a^{k} \equiv a^{k \mod (p - 1)} \pmod{p}.
  \end{equation*}
  Also:
  \begin{equation*}
    a^{p - 2} \equiv a^{-1} \pmod{p}.
  \end{equation*}
\item \textbf{Iterating over subsets.} Let \texttt{mask} be the binary representation
  of a set. Then \texttt{for (int i = mask; i != 0; i = (i - 1) \& mask)} will iterate
  over all the nonempty subsets of \texttt{mask}.
\item \textbf{Chinese remainder theorem.}
\item \textbf{Sum of harmonic series.}
  \begin{equation*} 
     \frac11 + \frac12 + \dots + \frac1n \in \mathcal{O}(\log n)
  \end{equation*}
\item \textbf{Number of primes below...}
  \begin{equation*}
    \begin{array}{rl}
       10^2 & 25 \\
       10^3 & 168 \\
       10^4 & 1229 \\
       10^5 & 9592 \\
       10^6 & 78498 \\
       10^7 & 664579
    \end{array}
  \end{equation*}
\end{itemize}

\end{document}
% !end_codebook
