\documentclass{article}
\usepackage[T1]{fontenc}
\usepackage[utf8]{inputenc}
\usepackage[english]{babel}
\usepackage{titlesec}
\usepackage{enumerate}
\usepackage{amsmath, amsfonts, amssymb, amsthm}
\usepackage{mathtools}
\usepackage{mathabx}
\usepackage{graphicx}
\usepackage{pdfpages}
\usepackage{forloop}

\numberwithin{equation}{section} % Number equations within sections (i.e. 1.1, 1.2, 2.1, 2.2 instead of 1, 2, 3, 4)
\numberwithin{figure}{section} % Number figures within sections (i.e. 1.1, 1.2, 2.1, 2.2 instead of 1, 2, 3, 4)
\numberwithin{table}{section} % Number tables within sections (i.e. 1.1, 1.2, 2.1, 2.2 instead of 1, 2, 3, 4)

\usepackage[hmarginratio=1:1,top=15mm, bottom=5mm, columnsep=0mm, left=6mm, right=3mm]{geometry} % Document margins
\usepackage[hang, small,labelfont=bf,up,textfont=it,up]{caption} % Custom captions under/above floats in tables or figures
\usepackage{booktabs} % Horizontal rules in tables


\usepackage{enumitem} % Customized lists
\setlist[itemize]{noitemsep} % Make itemize lists more compact
\usepackage{fancyhdr} % Proper control over header


\usepackage{minted}
\usepackage{etoolbox}
\AtBeginEnvironment{minted}{\fontsize{10}{10}\selectfont}
\usemintedstyle{emacs}
\titlespacing{\section}{10mm}{0mm}{0mm}



\title{\textbf{University of Tartu ICPC Team Notebook (2017-2018)}}
\date{\today}

\pagestyle{fancy}
\fancyhf{}
\lhead{University of Tartu}
\rhead{\thepage}
\setlength{\headsep}{0mm}

\binoppenalty=\maxdimen
\relpenalty=\maxdimen

\begin{document}

\begin{center}
	\smallskip
	\Large{\textbf{University of Tartu ICPC Team Notebook (2017-2018)}}

	\today
\end{center}
 
\tableofcontents

\par\noindent\rule{\textwidth}{0.4pt}

\section{gcc ordered set}
\begin{minted}[tabsize=2,baselinestretch=1,linenos,numbersep = 1mm, breaklines, frame=lines]{c++}
#include<ext/pb_ds/assoc_container.hpp>
#include<ext/pb_ds/tree_policy.hpp>
using namespace __gnu_pbds;
template <typename T>
using ordered_set = tree<T, null_type, less<T>, rb_tree_tag, tree_order_statistics_node_update>;
int main(){
	ordered_set<int>  cur;
	cur.insert(1); 
	cur.insert(3);
	cout << cur.order_of_key(2) << endl; // the number of elements in the set less than 2
	cout << *cur.find_by_order(0) << endl; // the 0-th smallest number in the set(0-based)
	cout << *cur.find_by_order(1) << endl; // the 1-th smallest number in the set(0-based)
}
\end{minted}
\section{Triangle centers}
\begin{minted}[tabsize=2,baselinestretch=1,linenos,numbersep = 1mm, breaklines, frame=lines]{c++}
const double min_delta = 1e-13;
const double coord_max = 1e6;
typedef complex < double > point;
point A, B, C; // vertixes of the triangle
bool collinear(){
	double min_diff = min(abs(A - B), min(abs(A - C), abs(B - C)));
	if(min_diff < coord_max * min_delta)
		return true;
	point sp = (B - A) / (C - A);
	double ang = M_PI/2-abs(abs(arg(sp))-M_PI/2); //positive angle with the real line
	return ang < min_delta;
}
point circum_center(){
	if(collinear())
		return point(NAN,NAN);
	//squared lengths of sides
	double a2, b2, c2;
	a2 = norm(B - C);
	b2 = norm(A - C);
	c2 = norm(A - B);
	//barycentric coordinates of the circumcenter
	double c_A, c_B, c_C;
	c_A = a2 * (b2 + c2 - a2);//sin(2 * alpha) may be used as well
	c_B = b2 * (a2 + c2 - b2);
	c_C = c2 * (a2 + b2 - c2);
	double sum = c_A + c_B + c_C;
	c_A /= sum;
	c_B /= sum;
	c_C /= sum;
	// cartesian coordinates of the circumcenter
	return c_A * A + c_B * B + c_C * C;
}
point centroid(){ //center of mass
	return (A + B + C) / 3.0;
}
point ortho_center(){ //euler line
	point O = circum_center();
	return O + 3.0 * (centroid() - O);
};
point nine_point_circle_center(){ //euler line
	point O = circum_center();
	return O + 1.5 * (centroid() - O);
};
point in_center(){
	if(collinear())
		return point(NAN,NAN);
	double a, b, c; //side lengths
	a = abs(B - C);
	b = abs(A - C);
	c = abs(A - B);
	//trilinear coordinates are (1,1,1)
	//barycentric coordinates
	double c_A = a, c_B = b, c_C = c;
	double sum = c_A + c_B + c_C;
	c_A /= sum;
	c_B /= sum;
	c_C /= sum;
	// cartesian coordinates of the incenter
	return c_A * A + c_B * B + c_C * C;
}
\end{minted}
\section{2D line segment}
\begin{minted}[tabsize=2,baselinestretch=1,linenos,numbersep = 1mm, breaklines, frame=lines]{c++}
const long double PI = acos(-1.0L);

struct Vec {
	long double x, y;
	
	Vec& operator-=(Vec r) {
		x -= r.x, y -= r.y;
		return *this;
	}
	Vec operator-(Vec r) {return Vec(*this) -= r;}
	
	Vec& operator+=(Vec r) {
		x += r.x, y += r.y;
		return *this;
	}
	Vec operator+(Vec r) {return Vec(*this) += r;}
	Vec operator-() {return {-x, -y};}
	Vec& operator*=(long double r) {
		x *= r, y *= r;
		return *this;
	}
	Vec operator*(long double r) {return Vec(*this) *= r;}
	Vec& operator/=(long double r) {
		x /= r, y /= r;
		return *this;
	}
	Vec operator/(long double r) {return Vec(*this) /= r;}
	
	long double operator*(Vec r) {
		return x * r.x + y * r.y;
	}
};
ostream& operator<<(ostream& l, Vec r) {
	return l << '(' << r.x << ", " << r.y << ')';
}
long double len(Vec a) {
	return hypot(a.x, a.y);
}
long double cross(Vec l, Vec r) {
	return l.x * r.y - l.y * r.x;
}
long double angle(Vec a) {
	return fmod(atan2(a.y, a.x)+2*PI, 2*PI);
}
Vec normal(Vec a) {
	return Vec({-a.y, a.x}) / len(a);
}
\end{minted}
\begin{minted}[tabsize=2,baselinestretch=1,linenos,numbersep = 1mm, breaklines, frame=lines]{c++}
struct Segment {
	Vec a, b;
	Vec d() {
		return b-a;
	}
};
ostream& operator<<(ostream& l, Segment r) {
	return l << r.a << '-' << r.b;
}

Vec intersection(Segment l, Segment r) {
	Vec dl = l.d(), dr = r.d();
	if(cross(dl, dr) == 0)
		return {nanl(""), nanl("")};
	
	long double h = cross(dr, l.a-r.a) / len(dr);
	long double dh = cross(dr, dl) / len(dr);
	
	return l.a + dl * (h / -dh);
}

//Returns the area bounded by halfplanes
long double getArea(vector<Segment> lines) {
	long double lowerbound = -HUGE_VALL, upperbound = HUGE_VALL;
	
	vector<Segment> linesBySide[2];
	for(auto line : lines) {
		if(line.b.y == line.a.y) {
			if(line.a.x < line.b.x)
				lowerbound = max(lowerbound, line.a.y);
			else
				upperbound = min(upperbound, line.a.y);
		}
		else if(line.a.y < line.b.y)
			linesBySide[1].push_back(line);
		else
			linesBySide[0].push_back({line.b, line.a});
	}
	
	sort(linesBySide[0].begin(), linesBySide[0].end(), [](Segment l, Segment r) {
		if(cross(l.d(), r.d()) == 0) return normal(l.d())*l.a > normal(r.d())*r.a;
		return cross(l.d(), r.d()) < 0;
	});
	sort(linesBySide[1].begin(), linesBySide[1].end(), [](Segment l, Segment r) {
		if(cross(l.d(), r.d()) == 0) return normal(l.d())*l.a < normal(r.d())*r.a;
		return cross(l.d(), r.d()) > 0;
	});
	
	//Now find the application area of the lines and clean up redundant ones
	vector<long double> applyStart[2];
	for(int side = 0; side < 2; side++) {
		vector<long double> &apply = applyStart[side];
		vector<Segment> curLines;
		
		for(auto line : linesBySide[side]) {
			while(curLines.size() > 0) {
				Segment other = curLines.back();
				
				if(cross(line.d(), other.d()) != 0) {
					long double start = intersection(line, other).y;
					if(start > apply.back())
						break;
				}
				
				curLines.pop_back();
				apply.pop_back();
			}
			
			if(curLines.size() == 0)
				apply.push_back(-HUGE_VALL);
			else
				apply.push_back(intersection(line, curLines.back()).y);
			curLines.push_back(line);
		}
		
		linesBySide[side] = curLines;
	}
	applyStart[0].push_back(HUGE_VALL);
	applyStart[1].push_back(HUGE_VALL);
	
	long double result = 0;
	{
		long double lb = -HUGE_VALL, ub;
		for(int i=0, j=0; i < (int)linesBySide[0].size() && j < (int)linesBySide[1].size();lb = ub) {
			ub = min(applyStart[0][i+1], applyStart[1][j+1]);
			
			long double alb = lb, aub = ub;
			Segment l0 = linesBySide[0][i], l1 = linesBySide[1][j];
			
			if(cross(l1.d(), l0.d()) > 0)
				alb = max(alb, intersection(l0, l1).y);
			else if(cross(l1.d(), l0.d()) < 0)
				aub = min(aub, intersection(l0, l1).y);
			alb = max(alb, lowerbound);
			aub = min(aub, upperbound);
			aub = max(aub, alb);
			
			{
				long double x1 = l0.a.x + (alb - l0.a.y) / l0.d().y * l0.d().x;
				long double x2 = l0.a.x + (aub - l0.a.y) / l0.d().y * l0.d().x;
				result -= (aub - alb) * (x1 + x2) / 2;
			}
			{
				long double x1 = l1.a.x + (alb - l1.a.y) / l1.d().y * l1.d().x;
				long double x2 = l1.a.x + (aub - l1.a.y) / l1.d().y * l1.d().x;
				result += (aub - alb) * (x1 + x2) / 2;
			}
			
			if(applyStart[0][i+1] < applyStart[1][j+1])
				i++;
			else
				j++;
		}
	}
	return result;
}
\end{minted}
\section{Dinic}
\begin{minted}[tabsize=2,baselinestretch=1,linenos,numbersep = 1mm, breaklines, frame=lines]{c++}
struct MaxFlow{
    typedef long long ll;
    const ll INF = 1e18;
    struct Edge{
        int u,v;
        ll c,rc;
        shared_ptr<ll> flow;
        pair<int,int> id() const {
            return make_pair(min(u,v),max(u,v));
        }
        Edge(int _u, int _v, ll _c, ll _rc = 0):u(_u),v(_v),c(_c),rc(_rc){
        }
        void join(const Edge &t){
            if(u == t.u){
                c += t.c;
                rc += t.rc;
            }
            else{
                c += t.rc;
                rc += t.c;
            }
        }
    };
    struct FlowTracker{
        shared_ptr<ll> flow;
        ll cap, rcap;
        bool dir;
        FlowTracker(ll _cap, ll _rcap, shared_ptr<ll> _flow, int _dir):cap(_cap),rcap(_rcap),flow(_flow),dir(_dir){ }
        ll rem() const {
            if(dir == 0){
                return cap-*flow;
            }
            else{
                return rcap+*flow;
            }
        }
        void add_flow(ll f){
            if(dir == 0)
                *flow += f;
            else
                *flow -= f;
            assert(*flow <= cap);
            assert(-*flow <= rcap);
        }
        operator ll() const { return rem(); }
        void operator-=(ll x){ add_flow(x); }
        void operator+=(ll x){ add_flow(-x); }
    };
    int source,sink;
    vector<vector<int> > adj;
    vector<vector<FlowTracker> > cap;
    vector<Edge> edges;
    MaxFlow(int _source, int _sink):source(_source),sink(_sink){
        assert(source != sink);
    }
    int add_edge(Edge e){
        edges.push_back(e);
        return edges.size()-1;

    }
    int add_edge(int u, int v, ll c, ll rc = 0){
        return add_edge(Edge(u,v,c,rc));
    }
    void group_edges(){
        map<pair<int,int>,vector<Edge> > edge_groups;
        for(auto edge: edges)
            if(edge.u != edge.v)
                edge_groups[edge.id()].push_back(edge);
        vector<Edge> grouped_edges;
        for(auto group: edge_groups){
            Edge main_edge = group.second[0];
            for(int i = 1; i < group.second.size(); ++i)
                main_edge.join(group.second[i]);
            grouped_edges.push_back(main_edge);
        }
        edges = grouped_edges;
    }
    vector<int> now,lvl;
    void prep(){
        int max_id = max(source,sink);
        for(auto edge : edges)
            max_id = max(max_id,max(edge.u,edge.v));
        adj.resize(max_id+1);
        cap.resize(max_id+1);
        now.resize(max_id+1);
        lvl.resize(max_id+1);
        for(auto &edge : edges){
            auto flow = make_shared<ll>(0);
            adj[edge.u].push_back(edge.v);
            cap[edge.u].push_back(FlowTracker(edge.c,edge.rc,flow,0));
            adj[edge.v].push_back(edge.u);
            cap[edge.v].push_back(FlowTracker(edge.c,edge.rc,flow,1));
            assert(cap[edge.u].back() == edge.c);
            edge.flow = flow;
        }
    }
    bool dinic_bfs(){
        fill(now.begin(),now.end(),0);
        fill(lvl.begin(),lvl.end(),0);
        lvl[source] = 1;
        vector<int> bfs(1,source);
        for(int i = 0; i < bfs.size(); ++i){
            int u = bfs[i];
            for(int j = 0; j < adj[u].size(); ++j){
                int v = adj[u][j];
                if(cap[u][j] > 0 && lvl[v] == 0){
                    lvl[v] = lvl[u]+1;
                    bfs.push_back(v);
                }
            }
        }
        return lvl[sink] > 0;
    }
    ll dinic_dfs(int u, ll flow){
        if(u == sink)
            return flow;
        while(now[u] < adj[u].size()){
            int v = adj[u][now[u]];
            if(lvl[v] == lvl[u] + 1 && cap[u][now[u]] != 0){
                ll res = dinic_dfs(v,min(flow,(ll)cap[u][now[u]]));
                if(res > 0){
                    cap[u][now[u]] -= res;
                    return res;
                }
            }
            ++now[u];
        }
        return 0;
    }
    ll calc(){
        prep();
        ll ans = 0;
        while(dinic_bfs()){
            ll cur = 0;
            do{
                cur = dinic_dfs(source,INF);
                ans += cur;
            }while(cur > 0);
        }
        return ans;
    }
};
int main(){
    int n,m;
    cin >> n >> m;
    auto mf = MaxFlow(1,n); // arguments source and sink, memory usage O(largest node index), sink doesn't need to be last
    int edge_index;
    for(int i = 0; i < m; ++i){
        int a,b,c;
        cin >> a >> b >> c;
        //undirected edge is a pair of edges (a,b,c,0) and (a,b,0,c)
        edge_index = mf.add_edge(a,b,c,c); //store edge index if care about flow value
    }
    mf.group_edges(); // small auxillary to remove multiple edges, only use this if we need to know TOTAL FLOW ONLY
    cout << mf.calc() << '\n';
    //cout << *mf.edges[edge_index].flow << '\n'; // ONLY if group_edges() WAS NOT CALLED
}
\end{minted}
\section{Min cost max flow O($\text{flow} \cdot n^2$)}
\begin{minted}[tabsize=2,baselinestretch=1,linenos,numbersep = 1mm, breaklines, frame=lines]{c++}
const int nmax=1055;
const ll inf=1e14;
int t, n, v; //0 is source, v-1 sink
ll rem_flow[nmax][nmax]; //set [x][y] for directed capacity from x to y.
ll cost[nmax][nmax]; //set [x][y] for directed cost from x to y. SET TO inf IF NOT USED
ll min_dist[nmax];
int prev_node[nmax];
ll node_flow[nmax];
bool visited[nmax];
ll tot_cost, tot_flow; //output
void min_cost_max_flow(){
	tot_cost=0;
	tot_flow=0;
	ll sink_pot=0;
	while(true){
		for(int i=0; i<=v; ++i){
			min_dist[i]=inf;
			visited[i]=false;
		}
		min_dist[0]=0;
		node_flow[0]=inf;
		int min_node;
		while(true){
			int min_node=v;
			for(int i=0; i<v; ++i){
				if((!visited[i]) && min_dist[i]<min_dist[min_node]){
					min_node=i;
				}
			}
			if(min_node==v){
				break;
			}
			visited[min_node]=true;
			for(int i=0; i<v; ++i){
				if((!visited[i]) && min_dist[min_node]+cost[min_node][i] < min_dist[i]){
					min_dist[i]=min_dist[min_node]+cost[min_node][i];
					prev_node[i]=min_node;
					node_flow[i]=min(node_flow[min_node], rem_flow[min_node][i]);
				}
			}
		}
		if(min_dist[v-1]==inf){
			break;
		}
		for(int i=0; i<v; ++i){
			for(int j=0; j<v; ++j){
				if(cost[i][j]!=inf){
					cost[i][j]+=min_dist[i];
					cost[i][j]-=min_dist[j];
				}
			}
		}
		sink_pot+=min_dist[v-1];
		tot_flow+=node_flow[v-1];
		tot_cost+=sink_pot*node_flow[v-1];
		int cur=v-1;
		while(cur!=0){
			rem_flow[prev_node[cur]][cur]-=node_flow[v-1];
			rem_flow[cur][prev_node[cur]]+=node_flow[v-1];
			cost[cur][prev_node[cur]]=0;
			if(rem_flow[prev_node[cur]][cur]==0){
				cost[prev_node[cur]][cur]=inf;
			}
			cur=prev_node[cur];
		}
	}
}
\end{minted}
\section{Aho Corasick O($|\text{alpha}|\sum{\text{len}}$)}
\begin{minted}[tabsize=2,baselinestretch=1,linenos,numbersep = 1mm, breaklines, frame=lines]{c++}
const int alpha_size=26;
struct node{
	node *nxt[alpha_size]; //May use other structures to move in trie
	node *suffix;
	node(){
		memset(nxt, 0, alpha_size*sizeof(node *));
	}
	int cnt=0;
};
node *aho_corasick(vector<vector<char> > &dict){
	node *root= new node;
	root->suffix = 0;
	vector<pair<vector<char> *, node *> > cur_state;
	for(vector<char> &s : dict)
		cur_state.emplace_back(&s, root);
	for(int i=0; !cur_state.empty(); ++i){
		vector<pair<vector<char> *, node *> > nxt_state;
		for(auto &cur : cur_state){
			node *nxt=cur.second->nxt[(*cur.first)[i]];
			if(nxt){
				cur.second=nxt;
			}else{
				nxt = new node;
				cur.second->nxt[(*cur.first)[i]] = nxt;
				node *suf = cur.second->suffix;
				cur.second = nxt;
				nxt->suffix = root; //set correct suffix link
				while(suf){
					if(suf->nxt[(*cur.first)[i]]){
						nxt->suffix = suf->nxt[(*cur.first)[i]];
						break;
					}
					suf=suf->suffix;
				}
			}
			if(cur.first->size() > i+1)
				nxt_state.push_back(cur);
		}
		cur_state=nxt_state;
	}
	return root;
}
//auxilary functions for searhing and counting
node *walk(node *cur, char c){ //longest prefix in dict that is suffix of walked string.
	while(true){
		if(cur->nxt[c])
			return cur->nxt[c];
		if(!cur->suffix){
			return cur;
		}
		cur = cur->suffix;
	}
}
void cnt_matches(node *root, vector<char> &match_in){
	node *cur = root;
	for(char c : match_in){
		cur = walk(cur, c);
		++cur->cnt;
	}
}
void add_cnt(node *root){ //After counting matches propagete ONCE to suffixes for final counts
	vector<node *> to_visit = {root};
	for(int i=0; i<to_visit.size(); ++i){
		node *cur = to_visit[i];
		for(int j=0; j<alpha_size; ++j){
			if(cur->nxt[j]){
				to_visit.push_back(cur->nxt[j]);
			}
		}
	}
	for(int i=to_visit.size()-1; i>0; --i){
		to_visit[i]->suffix->cnt += to_visit[i]->cnt;
	}
}
\end{minted}
\section{Suffix automaton O($(n+q)\log(|\text{alpha}|)$)}
\begin{minted}[tabsize=2,baselinestretch=1,linenos,numbersep = 1mm, breaklines, frame=lines]{c++}
class AutoNode {
 private:
  map< char, AutoNode * > nxt_char;  // Map is faster than hashtable and unsorted arrays
 public:
  int len; //Length of longest suffix in equivalence class.
  AutoNode *suf;
  bool has_nxt(char c) const { 
		return nxt_char.count(c); 
	}
  AutoNode *nxt(char c) {
    if (!has_nxt(c)) 
			return NULL;
    return nxt_char[c];
  }
  void set_nxt(char c, AutoNode *node) { 
		nxt_char[c] = node; 
	}
  AutoNode *split(int new_len, char c) {
    AutoNode *new_n = new AutoNode;
    new_n->nxt_char = nxt_char;
    new_n->len = new_len;
    new_n->suf = suf;
    suf = new_n;
    return new_n;
  }
  // Extra functions for matching and counting
  AutoNode *lower_depth(int depth) { //move to longest suffix of current with a maximum length of depth.
    if (suf->len >= depth) 
			return suf->lower_depth(depth);
    return this;
  }
  AutoNode *walk(char c, int depth, int &match_len) { //move to longest suffix of walked path that is a substring
    match_len = min(match_len, len);									//includes depth limit(needed for finding matches)
    if (has_nxt(c)) {																	//as suffixes are in classes match_len must be tracte eternally
      ++match_len;
      return nxt(c)->lower_depth(depth);
    }
    if (suf) 
			return suf->walk(c, depth, match_len);
    return this;
  }
  int paths_to_end = 0;
  void set_as_end() { //All suffixes of current node are marked as ending nodes. 
    paths_to_end = 1;
    if (suf) suf->set_as_end();
  }
  bool vis = false;
  void calc_paths_to_end() { 	//Call ONCE from ROOT. For each node calculates number of ways to reach an end node.
    if (!vis) {							 	//paths_to_end is ocurence count for any strings in current suffix equivalence class.
      vis = true;
      for (auto cur : nxt_char) {
        cur.second->calc_paths_to_end();
        paths_to_end += cur.second->paths_to_end;
      }
    }
  }
};
struct SufAutomaton {
  AutoNode *last;
  AutoNode *root;
  void extend(char new_c) {
    AutoNode *new_end = new AutoNode;  // The equivalence class containing the whole new string
    new_end->len = last->len + 1;
    AutoNode *suf_w_nxt = last;                        // The whole old string class
    while (suf_w_nxt && !suf_w_nxt->has_nxt(new_c)) {  // is turned into the longest suffix which
                                                       // can be turned into a substring of old state
                                                       // by appending new_c
      suf_w_nxt->set_nxt(new_c, new_end);
      suf_w_nxt = suf_w_nxt->suf;
    }
    if (!suf_w_nxt) {  // The new character isn't part of the old string
      new_end->suf = root;
    } else {
      AutoNode *max_sbstr = suf_w_nxt->nxt(new_c);  // Equivalence class containing longest
                                                    // substring which is a suffix of the new state.
      if (suf_w_nxt->len + 1 == max_sbstr->len) { // Check whether splitting is needed
        new_end->suf = max_sbstr;
      } else {
        AutoNode *eq_sbstr = max_sbstr->split(suf_w_nxt->len + 1, new_c);
        new_end->suf = eq_sbstr;
        // Make suffixes of suf_w_nxt point to eq_sbstr instead of mox_sbstr
        AutoNode *w_edge_to_eq_sbstr = suf_w_nxt;
        while (w_edge_to_eq_sbstr != 0 && w_edge_to_eq_sbstr->nxt(new_c) == max_sbstr) {
          w_edge_to_eq_sbstr->set_nxt(new_c, eq_sbstr);
          w_edge_to_eq_sbstr = w_edge_to_eq_sbstr->suf;
        }
      }
    }
    last = new_end;
  }
  SufAutomaton(string to_suffix) {
    root = new AutoNode;
    root->len = 0;
    root->suf = NULL;
    last = root;
    for (char c : to_suffix) extend(c);
  }
};
\end{minted}
\section{Templated multi dimensional BIT O($\log(n)^\text{dim}$)}
\begin{minted}[tabsize=2,baselinestretch=1,linenos,numbersep = 1mm, breaklines, frame=lines]{c++}
// Fully overloaded any dimensional BIT, use any type for coordinates, elements, return_value.
// Includes coordinate compression.
template < typename elem_t, typename coord_t, coord_t n_inf, typename ret_t >
class BIT {
  vector< coord_t > positions;
  vector< elem_t > elems;
  bool initiated = false;

 public:
  BIT() {
		positions.push_back(n_inf);
	}
  void initiate() {
    if (initiated) {
      for (elem_t &c_elem : elems)
				c_elem.initiate();
    } else {
      initiated = true;
      sort(positions.begin(), positions.end());
      positions.resize(unique(positions.begin(), positions.end()) - positions.begin());
      elems.resize(positions.size());
    }
  }
  template < typename... loc_form >
  void update(coord_t cord, loc_form... args) {
    if (initiated) {
      int pos = lower_bound(positions.begin(), positions.end(), cord) - positions.begin();
      for (; pos < positions.size(); pos += pos & -pos)
				elems[pos].update(args...);
    } else {
      positions.push_back(cord);
    }
  }
  template < typename... loc_form >
  ret_t query(coord_t cord, loc_form... args) { //sum in open interval (-inf, cord)
    ret_t res = 0;
    int pos = (lower_bound(positions.begin(), positions.end(), cord) - positions.begin())-1;
    for (; pos > 0; pos -= pos & -pos)
			res += elems[pos].query(args...);
    return res;
  }
};
template < typename internal_type >
struct wrapped {
  internal_type a = 0;
  void update(internal_type b) {
		a += b;
	}
  internal_type query() {
		return a;
	}
  // Should never be called, needed for compilation
  void initiate() {
		cerr << 'i' << endl; 
	}
  void update() { 
		cerr << 'u' << endl;
	}
};
int main() {
  // retun type should be same as type inside wrapped
  BIT< BIT< wrapped< ll >, int, INT_MIN, ll >, int, INT_MIN, ll > fenwick;
  int dim = 2;
  vector< tuple< int, int, ll > > to_insert;
	to_insert.emplace_back(1, 1, 1);
  // set up all positions that are to be used for update
  for (int i = 0; i < dim; ++i) {
    for (auto &cur : to_insert)
      fenwick.update(get< 0 >(cur), get< 1 >(cur));  // May include value which won't be used
    fenwick.initiate();
  }
  // actual use
  for (auto &cur : to_insert) 
		fenwick.update(get< 0 >(cur), get< 1 >(cur), get< 2 >(cur));
  cout << fenwick.query(2, 2)<<'\n';
}
\end{minted}
\section{Templated HLD}
\begin{minted}[tabsize=2,baselinestretch=1,linenos,numbersep = 1mm, breaklines, frame=lines]{c++}
class dummy {
public:
  dummy () {
  }
  
  dummy (int, int) {
  }

  void set (int, int) {
  }

  int query (int left, int right) {
    cout << this << ' ' << left << ' ' << right << endl;
  }
};

/* T should be the type of the data stored in each vertex;
 * DS should be the underlying data structure that is used to peform the
 * group operation. It should have the following methods:
 * * DS () - empty constructor
 * * DS (int size, T initial) - constructs the structure with the given size,
 *   initially filled with initial.
 * * void set (int index, T value) - set the value at index `index` to `value`
 * * T query (int left, int right) - return the "sum" of elements between left and right, inclusive.
 */
template<typename T, class DS>
class HLD {
  int vertexc;
  vector<int> *adj;
  vector<int> subtree_size;
  DS structure;
  DS aux;

  void build_sizes (int vertex, int parent) {
    subtree_size[vertex] = 1;
    for (int child : adj[vertex]) {
      if (child != parent) {
        build_sizes(child, vertex);
        subtree_size[vertex] += subtree_size[child];
      }
    }
  }

  int cur;
  vector<int> ord;
  vector<int> chain_root;
  vector<int> par;
  void build_hld (int vertex, int parent, int chain_source) {
    cur++;
    ord[vertex] = cur;
    chain_root[vertex] = chain_source;
    par[vertex] = parent;

    if (adj[vertex].size() > 1) {
      int big_child, big_size = -1;
      for (int child : adj[vertex]) {
        if ((child != parent) &&
            (subtree_size[child] > big_size)) {
          big_child = child;
          big_size = subtree_size[child];
        }
      }

      build_hld(big_child, vertex, chain_source);
      for (int child : adj[vertex]) {
        if ((child != parent) && (child != big_child)) {
          build_hld(child, vertex, child);
        }
      }
    }
  }
  
public:
  HLD (int _vertexc) {
    vertexc = _vertexc;
    adj = new vector<int> [vertexc + 5];
  }

  void add_edge (int u, int v) {
    adj[u].push_back(v);
    adj[v].push_back(u);
  }

  void build (T initial) {
    subtree_size = vector<int> (vertexc + 5);
    ord = vector<int> (vertexc + 5);
    chain_root = vector<int> (vertexc + 5);
    par = vector<int> (vertexc + 5);
    cur = 0;
    build_sizes(1, -1);
    build_hld(1, -1, 1);
    structure = DS (vertexc + 5, initial);
    aux = DS (50, initial);
  }

  void set (int vertex, int value) {
    structure.set(ord[vertex], value);
  }

  T query_path (int u, int v) { /* returns the "sum" of the path u->v */
    int cur_id = 0;
    while (chain_root[u] != chain_root[v]) {
      if (ord[u] > ord[v]) {
        cur_id++;
        aux.set(cur_id, structure.query(ord[chain_root[u]], ord[u]));
        u = par[chain_root[u]];
      } else {
        cur_id++;
        aux.set(cur_id, structure.query(ord[chain_root[v]], ord[v]));
        v = par[chain_root[v]];
      }
    }

    cur_id++;
    aux.set(cur_id, structure.query(min(ord[u], ord[v]), max(ord[u], ord[v])));

    return aux.query(1, cur_id);
  }

  void print () {
    for (int i = 1; i <= vertexc; i++) {
      cout << i << ' ' << ord[i] << ' ' << chain_root[i] << ' ' << par[i] << endl;
    }
  }
};

int main () {
  int vertexc;
  cin >> vertexc;

  HLD<int, dummy> hld (vertexc);
  for (int i = 0; i < vertexc - 1; i++) {
    int u, v;
    cin >> u >> v;

    hld.add_edge(u, v);
  }
  hld.build(0);
  hld.print();

  int queryc;
  cin >> queryc;
  for (int i = 0; i < queryc; i++) {
    int u, v;
    cin >> u >> v;

    hld.query_path(u, v);
    cout << endl;
  }
}
\end{minted}
\section{Templated Persitent Segment Tree}
\begin{minted}[tabsize=2,baselinestretch=1,linenos,numbersep = 1mm, breaklines, frame=lines]{c++}
template<typename T, typename comp>
class PersistentST {
  struct Node {
    Node *left, *right;
    int lend, rend;
    T value;

    Node (int position, T _value) {
      left = NULL;
      right = NULL;
      lend = position;
      rend = position;
      value = _value;
    }

    Node (Node *_left, Node *_right) {
      left = _left;
      right = _right;
      lend = left->lend;
      rend = right->rend;
      value = comp()(left->value, right->value);
    }

    T query (int qleft, int qright) {
      qleft = max(qleft, lend);
      qright = min(qright, rend);

      if (qleft == lend && qright == rend) {
        return value;
      } else if (qleft > qright) {
        return comp().identity;
      } else {
        return comp()(left->query(qleft, qright),
                      right->query(qleft, qright));
      }
    }
  };

  int size;
  Node **tree;
  vector<Node*> roots;
public:
  PersistentST () {
  }

  PersistentST (int _size, T initial) {
    for (int i = 0; i < 32; i++) {
      if ((1 << i) > _size) {
        size = 1 << i;
        break;
      }
    }
    
    tree = new Node* [2 * size + 5];

    for (int i = size; i < 2 * size; i++) {
      tree[i] = new Node (i - size, initial);
    }

    for (int i = size - 1; i > 0; i--) {
      tree[i] = new Node (tree[2 * i], tree[2 * i + 1]);
    }

    roots = vector<Node*> (1, tree[1]);
  }

  void set (int position, T _value) {
    tree[size + position] = new Node (position, _value);
    for (int i = (size + position) / 2; i >= 1; i /= 2) {
      tree[i] = new Node (tree[2 * i], tree[2 * i + 1]);
    }
    roots.push_back(tree[1]);
  }

  int last_revision () {
    return (int) roots.size() - 1;
  }

  T query (int qleft, int qright, int revision) {
    return roots[revision]->query(qleft, qright);
  }

  T query (int qleft, int qright) {
    return roots[last_revision()]->query(qleft, qright);
  }
};
\end{minted}
\section{FFT O($n\log(n)$)}
\begin{minted}[tabsize=2,baselinestretch=1,linenos,numbersep = 1mm, breaklines, frame=lines]{c++}
//Assumes a is a power of two
vector<complex<long double> > fastFourierTransform(vector<complex<long double> > a, bool inverse) {
	const long double PI = acos(-1.0L);
	int n = a.size();
	//Precalculate w
	vector<complex<long double> > w(n, 0.0L);
	w[0] = 1;
	for(int tpow = 1; tpow < n; tpow *= 2)
		w[tpow] = polar(1.0L, 2*PI * tpow/n * (inverse ? -1 : 1) );
	for(int i=3, last = 2;i<n;i++) {
		if(w[i] == 0.0L)
			w[i] = w[last] * w[i-last];
		else
			last = i;
	}
	
	//Rearrange a
	for(int block = n; block > 1; block /= 2) {
		int half = block/2;
		vector<complex<long double> > na(n);
		for(int s=0; s < n; s += block)
			for(int i=0;i<block;i++)
				na[s + half*(i%2) + i/2] = a[s+i];
		a = na;
	}
	
	//Now do the calculation
	for(int block = 2; block <= n; block *= 2) {
		vector<complex<long double> > na(n);
		int wb = n/block, half = block/2;
		
		for(int s=0; s < n; s += block)
			for(int i=0;i<half; i++) {
				na[s+i] = a[s+i] + w[wb*i] * a[s+half+i];
				na[s+half+i] = a[s+i] - w[wb*i] * a[s+half+i];
			}
		a = na;
	}
	
	return a;
}


struct Polynomial {
	vector<long double> a;
	
	long double& operator[](int ind) {
		return a[ind];
	}
	
	Polynomial& operator*=(long double r) {
		for(auto &c : a)
			c *= r;
		return *this;
	}
	Polynomial operator*(long double r) {return Polynomial(*this) *= r;}
	
	Polynomial& operator/=(long double r) {
		for(auto &c : a)
			c /= r;
		return *this;
	}
	Polynomial operator/(long double r) {return Polynomial(*this) /= r;}
	
	Polynomial& operator+=(Polynomial r) {
		if(a.size() < r.a.size())
			a.resize(r.a.size(), 0.0L);
		for(int i=0;i<(int)r.a.size();i++)
			a[i] += r[i];
		return *this;
	}
	Polynomial operator+(Polynomial r) {return Polynomial(*this) += r;}
	
	Polynomial& operator-=(Polynomial r) {
		if(a.size() < r.a.size())
			a.resize(r.a.size(), 0.0L);
		for(int i=0;i<(int)r.a.size();i++)
			a[i] -= r[i];
		return *this;
	}
	Polynomial operator-(Polynomial r) {return Polynomial(*this) -= r;}
	
	Polynomial operator*(Polynomial r) {
		int n = 1;
		while(n < (int)(a.size() + r.a.size() - 1) )
			n *= 2;
		
		vector<complex<long double> > fl(n, 0.0L), fr(n, 0.0L);
		for(int i=0;i<(int)a.size();i++)
			fl[i] = a[i];
		for(int i=0;i<(int)r.a.size();i++)
			fr[i] = r[i];
		
		fl = fastFourierTransform(fl, false);
		fr = fastFourierTransform(fr, false);
		
		vector<complex<long double> > ret(n);
		for(int i=0;i<n;i++)
			ret[i] = fl[i] * fr[i];
		ret = fastFourierTransform(ret, true);
		
		Polynomial result;
		result.a.resize(a.size() + r.a.size() - 1);
		for(int i=0;i<(int)result.a.size();i++)
			result[i] = ret[i].real() / n;
		return result;
	}
};
\end{minted}
\section{MOD int, extended Euctlidean}
\begin{minted}[tabsize=2,baselinestretch=1,linenos,numbersep = 1mm, breaklines, frame=lines]{c++}
pair<int, int> extendedEuclideanAlgorithm(int a, int b) {
	if(b == 0)
		return make_pair(1, 0);
	pair<int, int> ret = extendedEuclideanAlgorithm(b, a%b);
	return {ret.second, ret.first - a/b * ret.second};
}


struct Modint {
	static const int MOD = 1000000007;
	int val;
	
	Modint(int nval = 0) {
		val = nval;
	}
	
	Modint& operator+=(Modint r) {
		val = (val + r.val) % MOD;
		return *this;
	}
	Modint operator+(Modint r) {return Modint(*this) += r;}
	
	Modint& operator-=(Modint r) {
		val = (val + MOD - r.val) % MOD;
		return *this;
	}
	Modint operator-(Modint r) {return Modint(*this) -= r;}
	
	Modint& operator*=(Modint r) {
		val = 1LL * val * r.val % MOD;
		return *this;
	}
	Modint operator*(Modint r) {return Modint(*this) *= r;}
	
	Modint inverse() {
		int ret = extendedEuclideanAlgorithm(val, MOD).first;
		if(ret < 0)
			ret += MOD;
		return ret;
	}
	
	Modint& operator/=(Modint r) {
		return operator*=(r.inverse() );
	}
	Modint operator/(Modint r) {return Modint(*this) /= r;}
};
\end{minted}
\end{document}
