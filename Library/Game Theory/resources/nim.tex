\documentclass{article}
\usepackage{graphicx}
\usepackage{amsmath}

\title{Nim and Impartial Games}

\begin{document}
\maketitle
\begin{itemize}
\item
Applies to non-infinite, two-player Impartial Games (moves depend only on the position and not on who is moving)
\item
For simplicity assume loser is the player who can't make a move
\item
Let's define the nim-value for each node on the game graph as $g(v) = \min \{n \ge 0 : n \ne g(u) \text{ for } u \text{ in } \text{out}(v)\}$. For example: 
\begin{center}
\includegraphics[scale=0.6]{nim.png}
\end{center}
Note nodes with nim-value 0 are losing while everything else is winning
\item
When you have a large game consiting of making a single move in subgames $G_1, \ldots, G_k$, then if the state node $v$ corresponds to the node combination $v_1, \ldots, v_k$, then the nim value is 
\[
g(v) = g_1(v_1) \oplus \ldots \oplus g_k (v_k)
\]
Here $\oplus$ is bitwise xor. It's correct because:

\begin{enumerate}
\item
No next state has the same nim-value. First note that if the move was made in subgame $i$ to $v_i'$, then 
\begin{equation}
\begin{array}{lcl}
g(v') \oplus g(v) & = & [g_1(v_1) \oplus \ldots \oplus g_i(v_i') \oplus \ldots \oplus g_k(v_k)] \\
& & \oplus [g_1(v_1) \oplus \ldots \oplus g_i(v_i) \oplus \ldots \oplus g_k(v_k)] \\
& = & g_i(v_i') \oplus g_i(v_i) \\
\end{array}
\label{gamemove}
\end{equation}
Suppose by contradiction that there exists such move to $v'$ that $g(v) = g(v')$. In that case $g(v) \oplus g(v') = g_i(v_i') \oplus g_i(v_i) = 0 \Rightarrow g_i(v_i') = g_i(v_i)$ which contradicts the definition of $g_i$.

\item
This $g(v)$ is minimal. This can be proven by induction. For terminal nodes it's obviously true. Now let's look at some node $v$, and assume that for each connected node $v'$, the nim-value $g(v')$ is $g_1(v_1') \oplus \ldots \oplus g_k(v_k')$. Let's look at some non-negative value $a < g(v)$. If $g(v') = a$, then using \eqref{gamemove} for some $i$ we have:
\[
g_i(v_i) \oplus g_i(v_i') = a \oplus g(v) \Rightarrow g_i(v_i') = g_i(v_i) \oplus (a \oplus g(v))
\]
Since $a < g(v)$, in the leftmost bit $j$ where they differ, we must have $a_{(j)} = 0$ and $g(v)_{(j)} = 1$. Pick $i$ such that $g_i(v_i)_{(j)} = 1$. Since the leftmost 1-bit of $a \oplus g(v)$ is $j$, we have that: 
\[
g_i(v_i) \oplus (a \oplus g(v)) < g_i(v_i) \Rightarrow g_i(v_i') < g_i(v_i)
\]
Due to the definition of $g_i$, such $v_i'$ must exist. Since the game tree is a DAG, we can order the game-state nodes and easily apply induction.
\end{enumerate}

\end{itemize}
\end{document}