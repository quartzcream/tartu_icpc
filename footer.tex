% !begin_codebook
\section{Factsheet}
\includegraphics[scale=0.9]{comb.pdf}

\begin{itemize}
\item \textbf{Fermat's little theorem.} Let $p$ be prime. Then, for each integer $a$:
  \begin{equation*}
    a^{p - 1} \equiv 1 \pmod{p}.
  \end{equation*}
  Thus:
  \begin{equation*}
    a^{k} \equiv a^{k \mod (p - 1)} \pmod{p}.
  \end{equation*}
  Also:
  \begin{equation*}
    a^{p - 2} \equiv a^{-1} \pmod{p}.
  \end{equation*}
\item \textbf{Iterating over subsets.} Let \texttt{mask} be the binary representation
  of a set. Then \texttt{for (int i = mask; i != 0; i = (i - 1) \& mask)} will iterate
  over all the nonempty subsets of \texttt{mask}.
\item \textbf{Chinese remainder theorem.} We know that:
  \begin{align*}
    x &\equiv a_1 \pmod{n_1} \\
    x &\equiv a_2 \pmod{n_2}
  \end{align*}
  where $n_1$ and $n_2$ are (co)prime. We want to find $a_{1, 2}$ so that:
  \begin{equation*}
    x \equiv a_{1, 2} \pmod{n_1 \cdot n_2}.
  \end{equation*}
  A solution is given by:
  \begin{equation*}
    a_{1, 2} = a_1 m_2 n_2 + a_2 m_1 n_1,
  \end{equation*}
  where $m_1$ and $m_2$ are integers so that $m_1 n_1 + m_2 n_2 = 1$. Those
  values can be found using the Extended Euclidean algorithm.
\item \textbf{Sum of harmonic series.}
  \begin{equation*} 
     \frac11 + \frac12 + \dots + \frac1n \in \mathcal{O}(\log n)
  \end{equation*}
\item \textbf{Number of primes below...}
  \begin{equation*}
    \begin{array}{rl}
       10^2 & 25 \\
       10^3 & 168 \\
       10^4 & 1229 \\
       10^5 & 9592 \\
       10^6 & 78498 \\
       10^7 & 664579
    \end{array}
  \end{equation*}
\end{itemize}

\end{document}
% !end_codebook
