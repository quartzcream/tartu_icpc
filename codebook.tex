
\documentclass[a4paper]{article}
\usepackage[T1]{fontenc}
\usepackage[utf8]{inputenc}
\usepackage[english]{babel}
\usepackage{titlesec}
\usepackage{enumerate}
\usepackage{amsmath, amsfonts, amssymb, amsthm}
\usepackage{mathtools}
\usepackage{graphicx}
\usepackage{pdfpages}
\usepackage{forloop}
\usepackage{hyperref}
\usepackage{pdflscape}
\usepackage{multicol}

\usepackage{physics}
\usepackage{harpoon}
\renewcommand{\le}{\leqslant}
\renewcommand{\ge}{\geqslant}
\renewcommand{\epsilon}{\varepsilon}
\renewcommand{\phi}{\varphi}
\renewcommand{\theta}{\vartheta}
\newcommand{\Z}{\mathbb{Z}}
\newcommand{\N}{\mathbb{N}}
\newcommand{\Q}{\mathbb{Q}}
\newcommand{\R}{\mathbb{R}}
\newcommand{\C}{\mathbb{C}}
\newcommand{\F}{\mathbb{F}}
\newcommand{\qed}{\hfill$\blacksquare$}
\newcommand{\amp}{\, \& \,}
\newcommand{\imp}{\Rightarrow}
\renewcommand{\O}{\mathcal{O}}
\usepackage{listings}
\usepackage{xcolor}
\lstset{language=C++,
  aboveskip=0mm,belowskip=0mm,
  basicstyle=\footnotesize\ttfamily,
  keywordstyle=\color{blue}\ttfamily,
  stringstyle=\color{purple}\ttfamily,
  commentstyle=\color{red}\ttfamily,
  breaklines = true}
\renewcommand{\theenumi}{\alph{enumi}}
\DeclareMathOperator{\orient}{orient}
\DeclareMathOperator{\side}{side}
\DeclareMathOperator{\dist}{dist}
\DeclareMathOperator{\prp}{perp}
%\newcommand*{\vv}[1]{\overrightharp{\ensuremath{#1}}}
\newcommand{\vv}[1]{\overline{#1}}

\numberwithin{equation}{section} % Number equations within sections (i.e. 1.1, 1.2, 2.1, 2.2 instead of 1, 2, 3, 4)
\numberwithin{figure}{section} % Number figures within sections (i.e. 1.1, 1.2, 2.1, 2.2 instead of 1, 2, 3, 4)
\numberwithin{table}{section} % Number tables within sections (i.e. 1.1, 1.2, 2.1, 2.2 instead of 1, 2, 3, 4)

\usepackage[hmarginratio=1:1,top=19mm, bottom=10mm, columnsep=0mm, left=10mm, right=15mm]{geometry} % Document margins
\usepackage[hang, small,labelfont=bf,up,textfont=it,up]{caption} % Custom captions under/above floats in tables or figures
\usepackage{booktabs} % Horizontal rules in tables


\usepackage{enumitem} % Customized lists
\setlist[itemize]{noitemsep} % Make itemize lists more compact
\usepackage{fancyhdr} % Proper control over header


\usepackage{minted}
\usepackage{etoolbox}
\AtBeginEnvironment{minted}{\fontsize{10}{10}\selectfont}
\BeforeBeginEnvironment{minted}{\vspace*{-2.5mm}}
\AfterEndEnvironment{minted}{\vspace*{-4.5mm}}
\usemintedstyle{tartu_icpc}

\titleformat*{\section}{\bfseries}
\titlespacing*{\section}{10mm}{0ex}{0ex}

\title{\textbf{University of Tartu ICPC Team Notebook (2018-2019)}}
\date{\today}

\pagestyle{fancy}
\fancyhf{}
\lhead{University of Tartu}
\rhead{\thepage}
\setlength{\headsep}{0mm}

\hyphenpenalty=2000
\exhyphenpenalty=2000
\tolerance=2000
\emergencystretch=10pt
\binoppenalty=10
\relpenalty=10

\makeatletter
\renewcommand\tableofcontents{%
    \@starttoc{toc}%
}
\makeatother
\begin{document}

\begin{landscape}
\setlength\columnsep{10mm}
\begin{multicols*}{2}

\begin{center}
	\smallskip
  \Large{\textbf{University of Tartu ICPC Team Notebook (2018-2019)}}

	\today
\end{center}
 
\tableofcontents

\par\noindent\rule{\textwidth}{0.4pt}

% 
  \section{Maxflow Complexity}
  \begin{description}[align=left]
    \item [$\mathcal{O}(V^2E)$] -- Dinic
    \item [$\Theta({VE \log U})$] -- Capacity scaling
    \item [$\Theta(\text{flow}E)$] -- Small flow
    \item [$\Theta(\min\{V^{\frac{2}{3}},E^{\frac{1}{2}}\}E)$] -- Unitary capacities
    \item [$\Theta(\sqrt{V}E)$] -- Each vertex other than S,T has only a single incoming unitary edge or outgoing one (bipartite matching)
    \item [$\Theta(\text{flow}E \log V)$] -- Min-cost-max flow
  \end{description}
  \section{Min Rotation of string}
  \begin{lstlisting}
    int a=0, N=s.size();
    s += s;
    ran(b,0,N){
      ran(i,0,N) {
        if (a+i == b || s[a+i] < s[b+i]) {
          b += max(0, i-1);
          break;
        }
        if (s[a+i] > s[b+i]) {
          a = b;
          break;
        }
      }
    }
    return a;
  \end{lstlisting}
\end{multicols}
\unboldmath
\setlength\columnsep{10mm}
\begin{multicols*}{2}
% 
\section{Symmetric Submodular Functions; Queyrannes's algorithm}

\noindent {\bf SSF}: such function $f : V \rightarrow R$ that satisfies $f(A) = f(V/A)$
and for all $x \in V, X \subseteq Y \subseteq V$ it holds that
$f(X+x) - f(X) \leq f(Y+x) - f(Y)$.

\noindent {\bf Hereditary family}: such set $I \subseteq 2^V$ so that $X \subset Y \wedge Y \in I \Rightarrow X \in I$.

\noindent {\bf Loop}: such $v \in V$ so that ${v} \notin I$.

\begin{minted}[tabsize=2,baselinestretch=1,linenos,numbersep = 1mmbreaklines, frame=lines, texcomments=true, mathescape=true]{python}
def minimize():
	s = merge_all_loops()
	while size >= 3:
		t, u = find_pp()
		{u} is a possible minimizer
		tu = merge(t, u)
		if tu not in I:
			s = merge(tu, s)
	for x in V:
		{x} is a possible minimizer
def find_pp():
	W = {s} # s as in minimizer()
	todo = V/W
	ord = []
	while len(todo) > 0:
		x = min(todo, key=lambda x: f(W+{x}) - f({x}))
		W += {x}
		todo -= {x}
		ord.append(x)
	return ord[-1], ord[-2]

def enum_all_minimal_minimizers(X): 
  # X is a inclusionwise minimal minimizer
	s = merge(s, X)
	yield X
	for {v} in I:
		if f({v}) == f(X):
			yield X
			s = merge(v, s)
	while size(V) >= 3:
		t, u = find_pp()
		tu = merge(t, u)
		if tu not in I:
			s = merge(tu, s)
		elif f({tu}) = f(X):
			yield tu
			s = merge(tu, s)
\end{minted}

% 
\end{multicols}
\boldmath
\setlength\columnsep{5mm}
\begin{multicols}{3}
\section{2D geometry}
\noindent
Define $\orient(A, B, C) = \vv{AB} \times \vv{AC}$. CCW iff $>0$. \\
Define $\prp((a, b)) = (-b, a)$. The vectors are orthogonal. \\
For line $ax + by = c$ def $\vv{v} = (-b, a)$. \\
Line through $P$ and $Q$ has $\vv{v} = \vv{PQ}$ and $c = \vv{v} \times P$. \\
$\side_l (P) = \vv{v_l} \times P - c_l$ sign determines which side $P$ is on from $l$. \\
$\dist_l (P) = \side_l (P) / \norm{v_l}$ squared is integer. \\
Sorting points along a line: comparator is $\vv{v} \cdot A < \vv{v} \cdot B$. \\
Translating line by $\vv{t}$: new line has $c' = c + \vv{v} \times \vv{t}$. \\
Line intersection: is $(c_{l} \vv{v}_m - c_m \vv{v}_l) / (\vv{v}_l \times \vv{v}_m)$. \\
Project $P$ onto $l$: is $P - \prp(v) \side_l(P) / \norm{v}^2$. \\
Angle bisectors: $\vv{v} = \vv{v}_l / \norm{\vv{v}_l} + \vv{v}_m / \norm{\vv{v}_m}$ \\
$c = c_l / \norm{\vv{v}_l} + c_m / \norm{\vv{v}_m}$. \\
$P$ is on segment $AB$ iff $\orient(A, B, P) = 0$ and $\vv{PA} \cdot \vv{PB} \le 0$. \\
Proper intersection of $AB$ and $CD$ exists iff $\orient(C, D, A)$ and $\orient(C, D, B)$
have opp. signs and $\orient(A, B, C)$ and $\orient(A, B, D)$ have opp. signs. Coordinates: \\
\begin{equation*}
  \frac{A \orient(C, D, B) - B \orient(C, D, A)}{\orient(C, D, B) - \orient(C, D, A)}.
\end{equation*}
Circumcircle center:
\lstinputlisting[firstline = 1, lastline = 4]{geometry_snippets.cpp}
Circle-line intersect:
\lstinputlisting[firstline = 7, lastline = 14]{geometry_snippets.cpp}
Circle-circle intersect:
\lstinputlisting[firstline = 17, lastline = 25]{geometry_snippets.cpp}
Tangent lines:
\lstinputlisting[firstline = 28, lastline = 36]{geometry_snippets.cpp}
\section{3D geometry}
\noindent
$\orient(P, Q, R, S) = (\vv{PQ} \times \vv{PR}) \cdot \vv{PS}$. \\
  $S$ above $PQR$ iff $>0$. \\
For plane $ax + by + cz = d$ def $\vv{n} = (a, b, c)$. \\
Line with normal $\vv{n}$ through point $P$ has $d = \vv{n} \cdot P$. \\
$\side_\Pi (P) = \vv{n} \cdot P - d$ sign determines side from $\Pi$. \\
$\dist_\Pi (P) = \side_\Pi (P) / \norm{\vv{n}}$. \\
Translating plane by $\vv{t}$ makes $d' = d + \vv{n} \cdot \vv{t}$. \\
Plane-plane intersection of has direction $\vv{n}_1 \times \vv{n}_2$ and
goes through $((d_1 \vv{n}_2 - d_2 \vv{n}_1) \times \vv{d}) / \norm{\vv{d}}^2$. \\
Line-line distance:
\lstinputlisting[firstline = 39, lastline = 43]{geometry_snippets.cpp}
Spherical to Cartesian: \\ $(r \cos \phi \cos \lambda, r \cos \phi \sin \lambda, r \sin \phi)$. \\
Sphere-line intersection:
\lstinputlisting[firstline = 46, lastline = 52]{geometry_snippets.cpp}
Great-circle distance between points $A$ and $B$ is $r \angle AOB$. \\
Spherical segment intersection:
\lstinputlisting[firstline = 55, lastline = 87]{geometry_snippets.cpp}
Angle between spherical segments $AB$ and $AC$ is angle between $A \times B$ and $A \times C$. \\
Oriented angle: subtract from $2\pi$ if mixed product is negative. \\
Area of a spherical polygon:
\begin{equation*}
  r^2 [ \text{sum of interior angles} - (n - 2)\pi ].
\end{equation*}
% 
\includepdf[pages={-}, angle=90,pagecommand={\pagestyle{fancy}}]{codebookpart2}
\end{document}
% 